% -*- TeX-master: "YFT2023.tex"; eval: (longlines-mode); fill-column: 100000 -*-

\section{Data compilation}
\label{sec:data_compilation}

\subsection{General notes}
\label{sec:general_notes}

Data used in the yellowfin tuna stock assessment using MFCL consist of catch, effort, length \& weight-frequency data for the fisheries defined in the analysis, and tag-recapture data. Conditional age-at-length data are also used directly as data in the assessment model, as was recommended by the peer review of the 2020 yellowfin tuna assessment \citep{punt_independent_2023}. Improvements in these data inputs are ongoing and readers should refer to the companion papers highlighted at the end of \autoref{sec:introduction} for detailed descriptions of how the data and biological inputs were formulated as only brief overviews are provided below. A summary of the data available for the assessment is provided in \autoref{fig:data_coverage_tile}.

\subsection{Spatial stratification}
\label{sec:spatial_stratification}

The geographical area considered in the assessment corresponds to the WCPO (from 50\degree{}N to 40\degree{}S between 120\degree{}E and 150\degree{}W) and oceanic waters adjacent to the east Asian coast (110\degree{}E between 20\degree{}N and 10\degree{}S). The eastern boundary of the assessment excludes the WCPFC Convention area component that overlaps with the Inter American Tropical Tuna Commission (IATTC) area. We began the stepwise model development with the previous 9 region model structure \citep{vincent_stock_2020}(\autoref{fig:map_5regions_9regions}), but as we progressed through the stepwise model development, aspects of model convergence began to deteriorate and a decision was made to implement the 5 region structure which had better convergence properties, including a positive definite Hessian, which was indicated as being essential for diagnostic models by SC18. The 5 region stratification was supported by the review paper \citep{hamer_review_2023}, and we maintained the fisheries definitions for the extraction fisheries as applied in the 9 region model. That is, gear/flag specific fisheries that were defined by separate regions in the 9 region stratification, remained defined as separate fisheries within the larger regions of the simplified stratification, noting that the simplified stratification involved merging regions of the 9 region structure rather than altering boundaries and creating entirely new regions. This is akin to a fleets-as-areas approach within the larger simplified regions. Maintaining the extraction fisheries definitions was partly for efficiency and partly to maintain the fishery definitions rather that changing both the spatial stratification and fishery definitions together. In this way the effects of simplified stratification could be isolated. This also accounts for heterogeneities in fisheries composition data such as for the region around Hawaii. The 5 region structure maintains the region around the Papua New Guinea and Solomon Islands area to accommodate the longer residency of yellowfin in these archipelagic waters and the tagging data and associated mixing period assumptions.

While the 5 region structure was preferable over the 9 region structure in terms of model performance, and provides what we feel is a suitable spatial structure, with more time we may have been able to improve the performance or the 9 region model. We suggest a stand alone project is required to fully explore and compare the benefits and limitations of alternative spatial structures, with review by SC and their advice on a preferred option for future assessments.

Readers should be aware of the differences in the region numbering between the 5 and 9 region structures, for example region 8 (Papua New Guinea/Solomon Islands) in the 9 region structure is region 3 in the 5 region structure. When model region numbers are referred to they relate to the 5 region structure unless otherwise specified.

\subsection{Temporal stratification}
\label{sec:temporal_stratification}

The time period covered by the assessment is 1952--2021 which includes all significant post-war tuna fishing in the WCPO. Within this period, data were compiled into quarters (1; Jan--Mar, 2; Apr--Jun, 3; Jul--Sep, 4; Oct--Dec). As agreed at SC12, the assessment does not include data from the most recent calendar year as this is considered incomplete at the time of formulating the assessment inputs. Recent year data are also often subject to significant revision post-SC, in particular the longline data on which this assessment greatly depends.

\subsection{Definition of fisheries}
\label{sec:fisheries_definitions}

MFCL requires \enquote{fisheries} to be defined that consist of relatively homogeneous fishing units. Ideally, the defined fisheries will have selectivity and catchability characteristics that do not vary greatly over time and space. For most pelagic fisheries assessments, fisheries are typically defined according to combinations of gear type, fishing method and region, and for some, also flag or fleet. There are 41 fisheries defined for both the 5 and 9 region models used in this assessment (\autoref{tab:fisheries_definitions}) consisting of two fishery types: ``index fisheries'', that are used for generating indices of abundance (see further below), and ``extraction fisheries'' that account for the catches removed from the stock. Extraction fisheries include longline, purse seine, pole and line and various miscellaneous fisheries in the Indonesia/Philippines/Vietnam region. The fisheries definitions for the 2023 assessment are consistent with those used in the 2020 9 region assessment, but region numbers change to account for the 5 region structure. A graphical summary of the availability of data for each fishery used in the assessment model is provided in \autoref{fig:data_coverage_tile}.

Equatorial purse seine fishing activity was aggregated over all nationalities, but stratified by region and set type, in order to sufficiently capture the variability in fishing operations and selectivity of different purse seine set types. Set types were grouped into associated (i.e. log, FAD, whale, dolphin, and unknown set types) and unassociated (free-school) sets. Additional fisheries were defined for pole-and-line fisheries and miscellaneous fisheries (gillnets, ringnets, hook-and-line, handlines etc.) in the western equatorial area. At least one longline index fishery was defined in each region, although in regions 2 and 4 extraction longline fishing was separated into distant water and offshore components to account for the apparent differences in fishing practices and selectivity for these fleets in these regions.

\textbf{Index fisheries}: The catch-conditioned approach (\autoref{sec:catch_cond}) allows the specification  of ``index fisheries'' that are used to provide standardised CPUE indices of abundance for each model region. Index fisheries are akin to ``survey fisheries'' as described for other software such as Stock Synthesis, and may be the same fisheries as the extraction fisheries, but when used as index fisheries they do not take any catch, and must have effort data to allow modelling of CPUE. For this assessment one index fishery is defined for each model region as a composite fishery composed of all longline fisheries operating in each assessment region \citep{teears_cpue_2023}. Index fisheries may be grouped if it is felt that the CPUE reflects differences in average abundance among regions. For this assessment, index fisheries are grouped which allows the standardised CPUE to provide information on regional as well as temporal relative abundance. The full longline operational dataset, described in \cite{mckechnie_analysis_2015}, \cite{ducharme-barth_analysis_2020}, and \cite{teears_cpue_2023}, was used as the basis for developing the index fisheries CPUE. The CPUE standardisation approach for the index fisheries is described in detail in \cite{teears_cpue_2023}, see further \autoref{sec:catch_effort_LL}.

The standardized indices for each region are scaled by the regional scaling factors derived from the geostatistical CPUE standardization model. Catchability for the index fisheries is then assumed to be constant over time and shared across the assessment regions in order to scale the population. This means that the assessment model estimates relative abundance among spatial strata that is generally similar to the scaled CPUE relative abundance. The regular longline extraction fisheries are based on the same data set, but are disaggregated into the longline fisheries defined in \autoref{tab:fisheries_definitions}.

The size composition data (length and weight-frequency) for the extraction fisheries is assumed to represent the actual composition of the removed fish for any space-time strata, and in the data preparation process are weighted by the catch in order to represent the fisheries extractions at the spatial (region) and temporal (quarter) resolution of the model \citep{peatman_analysis_2023}. However, for the index fisheries, while the same aggregation process is conducted, the size data are weighted by standardised CPUE (rather than by catch) so that the size data are more representative of the abundance of the underlying population in each region and time period. Further, because the size data for the index and extraction fisheries are effectively being used twice (but weighted differently), the likelihood weighting for the size composition is adjusted such that the original intended weight (effective sample size) in the likelihood is preserved.

\subsection{Catch and effort data}
\label{sec:catch_effort_data}

\subsubsection{General characteristics}

Catch and effort data were compiled according to the fisheries defined in \autoref{tab:fisheries_definitions}. Catches by the longline fisheries were expressed in numbers of fish, and catches for all other fisheries expressed in weight (mt). This is consistent with the form in which the catch data are recorded and reported for these fisheries. The catches are aggregated at 5\degree{} x 5\degree{} and quarterly resolution, with the aggregation process either conducted by SPC, where operational data is available to inform this, or by the particular countries following statistical procedures that are reported to the Commission. For some fisheries, notably those in region 2 - Indonesian/Philippines/Vietnam - operational information on quarterly or spatial patterns in catches is poor so the annual catches are aggregated evenly across quarters and spatial cells. This is done by SPC.

In the catch-conditioned model, effort is not essential but is required (at least for a recent period of time) for projection analyses involving fisheries managed under effort rather than catch controls. The effort data are necessary to derive recent estimates of catchability for running the effort based projections. In this case the main industrial purse seine fisheries operating in the tropical region (i.e., regions 2, 3, 4) are managed under effort control. Effort data for these purse seine fisheries are defined as number of sets specified by set type (associated or unassociated), and are included for the last 12 quarters to facilitate projections. The period of 12 quarters is consistent with previous projections using catch errors models. For this assessment several other fisheries also have effort included to allow effort based projection for management purposes, these are the longline extraction fisheries, with effort measured as numbers of hooks per set, and the Japanese pole and line fishery with effort measured in vessel fishing days.

Total annual catches by major gear categories for the WCPO are shown in \autoref{fig:catch_hist_full} and a regional breakdown is provided in \autoref{fig:catch_hist_regional}. Catches by fishery groups are provided in \autoref{fig:catch_by_fishery_longline}, \autoref{fig:catch_by_fishery_ps} and \autoref{fig:catch_by_fishery_other}. The spatial distribution of catches over the past ten years is provided in \autoref{fig:catch_map}. Discarded catches are estimated to be minor and were not included in the analysis. Catches in the northern region are low and highly seasonal and the annual catch has been relatively stable over much of the assessment period. Most of the catch occurs in the tropical regions (2, 3, and 4).

A number of significant trends in the fisheries have occurred over the model period, specifically:

\begin{itemize}
  \item The steady increase in total yellowfin catch over most of the assessment period, with the highest overall catches reported in the most recent years.
  \item The steady increase in catch for the domestic fisheries of Indonesia and the Philippines (region 2) since 1970, where mostly small juveniles are taken, and more significant increase in the catches over the last 15 years. Some of this trend can be related to improved information to estimate catches.
  \item The relatively stable and low catches of yellowfin in the northern and southern temperate regions by longline vessels (regions 1, and 5).
  \item The development of the equatorial purse-seine fisheries from the mid-1970s, and corresponding increased catches, particularly in equatorial regions, with the purse seine catch recently at 3-5 times higher than the longline catch.
  \item Large changes in the purse seine fleet composition and the increasing size and likely efficiency of the fleet.
\end{itemize}

\subsubsection{Purse seine}
\label{sec:catch_effort_PS}

For the industrial purse seine fisheries predominantly operating in tropical regions 3, 4 and 8, catch by species within each set type (associated or unassociated) is determined by applying estimates of species composition from observer-collected samples to total catches estimated from raised logsheet data \citep{hampton_annual_2016,peatman_project_2021,peatman_project_2023}. For the Japanese (JP) fleet for which there is greater confidence in species-based reporting, reported catch by species is used. Purse seine catch for Philippines (PH) and Indonesian (ID) domestic purse seine fisheries, predominantly operating in Region 7, was derived from raised port sampling data provided by these countries. We note that the COVID-19 pandemic resulted in low observer coverage of the purse seine fleets for the last two years of the assessment period. The implications of the low observer coverage on the purse seine catch composition estimates could not be fully explored under the time constraints, but preliminary analysis suggest the estimates have been relatively robust to the lower observer coverage \citep{hamer_report_2023}.

\subsubsection{Longline}
\label{sec:catch_effort_LL}

For the longline fisheries catches in number of fish by species are derived from raised logbook data or annual catch estimates provided by specific countries. Effort is in terms of hooks per set.

The longline CPUE indices are one of the most important inputs to the assessment as they provide indices of abundance over time for each region. The CPUE indices are implemented as \enquote{index} fisheries where they are assumed to have the same catchability and are grouped to provide information on biomass scaling among the model regions.

The index fishery CPUE time series for the 2023 assessment were derived from the operational longline dataset for the Pacific region. This dataset is an amalgamation of operational level data from the distant-water fishing nations (DWFN), United States, Australian, New Zealand and Pacific Island countries and territories (PICTs) longline fleets operating in the Pacific basin. It represents the most complete spatiotemporal record of longline fishing activity in the Pacific, spanning from 1952 through to the present and is the result of collaborative ongoing data-sharing efforts from many countries. This dataset was first created in 2015 in support of the Pacific-wide bigeye tuna stock assessment \citep{mckechnie_analysis_2015}, and was subsequently analyzed to generate indices of relative abundance for the 2017 and 2020 WCPFC bigeye and yellowfin tuna stock assessments \citep{mckechnie_stock_2017,ducharme-barth_analysis_2020}. Since 2017 spatiotemporal approaches have been used for CPUE modeling in WCPFC stock assessments \citep{tremblay-boyer_stock_2017,ducharme-barth_analysis_2020}. For this assessment we build on these previous efforts and have transitioned from using the VAST software \citep{thorson_guidance_2019,thorson_geostatistical_2015} for these analyses to using the sdmTMB package \citep{anderson_sdmtmb_2022}. sdmTMB was preferred over VAST due to it's greater computational efficiency, ease of use, and the ready availability of online support from a larger user community than VAST.

A detailed description of the methods for generating the spatiotemporal abundance indices is provided in \citet{teears_cpue_2023}. Briefly, it was first confirmed that the sdmTMB package could closely replicate the previous VAST indices using the data from the 2020 assessment. After this step a model was run with an increased density of mesh knots (371 versus 154) and the same spatiotemporal subsampling design as the previous assessment. Following this, further exploration of alternative models was conducted considering additional covariates in addition to those applied in the 2020 assessment. These included density covariates of SST, depth of the 15° C isotherm, and difference between the depth of the 12° C isotherm and the 18° C isotherm. As per the previous assessment, catchability covariates of hooks between floats (HBF) and vessel FLAG were included. A vessel ID covariate was considered, but there were over 6,000 unique vessel IDs and this was not considered computationally feasible. El Niño Southern Oscillation data were also included as a potential covariate but caused model instability and therefore, this was not included in the analyses. A model selection process described in \citet{teears_cpue_2023} was followed and the final model for yellowfin included HBF, vessel FLAG, depth of 15° C isotherm, and the difference between the depth of the 12° C  isotherm and the 18° C isotherm.

In response to the yellowfin peer review, two additional analyses were conducted. One analysis involved running separate models for northern, equatorial, and southern regions with `non-viable' (poorly sampled) 5° $\times$ 5° grid cells removed and comparing the predictions to the results of the same aggregated northern, equatorial, and southern regions from the Pacific-wide indices. Results indicated differences in spatial characterization, although the differences were in areas with comparatively low abundance and had limited implications. An analysis was also conducted comparing a principal-fleet model (Japanese fleet only) to the multi-fleet results to assess the effects of combining fleets. The indices derived from multiple fleets were very similar to the principal-fleet results. It was decided that the outcomes of these additional analyses did not warrant changing the initial approach (see \citet{teears_cpue_2023}).

\subsubsection{Other fisheries}
\label{sec:catch_effort_other}

Effort data for the ID, PH, and VN surface fisheries and Japanese research longline fisheries are unavailable. However, as these fisheries are not part of the index fisheries, the catch-conditioned approach does not require effort data for these extraction fisheries. Catch estimates for the ID/PH/VN fisheries are derived from various port sampling programmes dating back to the 1960s for ID and the PH, and early 2000s for VN \citep{williams_overview_2023}.

\subsection{Size data}
\label{sec:size_data}

Available length-frequency data for each of the fisheries were compiled into 95 x 2cm size classes from 10--12 cm to 198--200 cm. Weight data were compiled into 200 x 1 kg size classes from 0--1 kg to 199--200 kg. Most weight data were recorded as processed weights (usually recorded to the nearest kilogram). Processing methods varied between fleets requiring the application of fishery-specific conversion factors to convert the available weight data to whole fish equivalents. Details of the conversion to whole weight are described in \citet{macdonald_project_2023}. Data were either collected onboard by fishers, through observer programs, or through port sampling. Each size-frequency record in the model consisted of the actual number of yellowfin tuna measured and \autoref{fig:data_coverage_tile} provides details of the temporal availability of length and weight-frequency data (also see \citealp{teears_cpue_2023}). Note that a maximum effective sample size of 1,000 is implemented in MFCL when using the robust normal likelihood for size composition data. The effective sample size was further down-weighted as explained in \autoref{sec:sizefreq_likelihood}. Summaries of the available size composition data by year and fishery are provided in \autoref{fig:length_comp_sampl_sum} and \autoref{fig:size_comp_sampl_sum}.

\subsubsection{Purse seine}
\label{sec:size_PS}

Only length-frequency samples are used in the yellowfin assessment for purse seine fisheries. Prior to 2014, the assessments used only observer samples which had been corrected for grab-sample bias. As observer coverage had been very low and unrepresentative in early years, there were many gaps and the time series of size data did not show evidence of modal progression. Two major changes were implemented for the 2014 assessment and are described in detail in \citet{abascal_analysis_2014}: first the long time series of port sampling data from Pago Pago was included, and second all samples were weighted by the catch - both at the set and strata level, with thresholds applied to ensure that small samples from important catch strata did not get too much weight (consistent with the approach taken for the longline fishery). The pre-processing of the purse seine length composition data for the current assessment is described in \citet{peatman_analysis_2023}. Length-frequency data were unavailable for the \enquote{all flags} associated purse seine fishery in region 2 (Fishery 30). In the model, it was assumed to share a selectivity with the \enquote{all flags} associated purse seine fishery in the adjacent region 4 (Fishery 13).

\subsubsection{Longline}
\label{sec:size_LL}

A review of all available longline length and weight-frequency data for yellowfin was undertaken by \citet{mckechnie_analysis_2014}. Details on the data and analytical approach used to construct the size data inputs for the current assessment are in \citet{peatman_analysis_2023} and \citet{teears_cpue_2023}. The key principle used in constructing the size composition inputs was not to use weight and length data at the same time, even if it was available, as it would either introduce conflict (if data were in disagreement) or over-weight the model fit (if they were in agreement). The general approach used in previous assessments for the \enquote{extraction} fisheries was that weight-frequency samples should be weighted with respect to the spatial distribution of flag-specific catch within each region. This is done so that catch is extracted from the population at the appropriate size and is not biased by issues such as small catches with lots of weight frequency samples. Weight-frequency data were used over length frequency based on the spatiotemporal coverage and number of samples. However, despite additional weight frequency data being provided by Japan for the 2020 assessment, the number of available weight-frequency samples has declined in recent years. The 2020 assessment conducted a sensitivity analysis involving switching from weight to length-frequency data for the longline fisheries in regions 4, 5, and 6 of the 9 region structure beginning in 2000. The results were relatively insensitive to this change. We suggest that the next assessment could develop a longline size composition data set that optimises the use of both length and weight frequency data with respect of maximising spatial and temporal coverage, and transitioning to length composition data for the recent years.

Size composition data were prepared similarly for the index fisheries \citep{peatman_analysis_2023}. The approach for the index fisheries differed from the one briefly described above for the extraction fisheries in that the size-frequency samples were weighted with respect to the spatial distribution of abundance as predicted by the spatiotemporal CPUE standardization model \citep{teears_cpue_2023}. This is to allow size compositions to inform temporal variation in population abundance and size. To generate the size composition data for the index fisheries, data were first subset to match the nationalities of the \enquote{all flags} longline fisheries in each region. This was done to prevent shifts in size composition as a result of a change in sampling between fisheries.

Given that the same data were used for both the extraction and index fisheries, the observed number of size-frequency samples input into the assessment was divided by 2 for both the extraction and index fisheries. The maximum effective sample size in the stock assessment model was also divided by two for these fisheries (i.e. 500 as opposed to the default value of 1,000 assumed for the other fisheries).

\subsubsection{Other fisheries}
\label{sec:size_others}

Size composition data for the Philippines domestic fisheries, both small-fish fisheries (Fishery 17) and large-fish handline fisheries (Fishery 18), were derived from a number of port sampling programs conducted in the Philippines since the 1980s. In more recent years, size-sampling data have been substantially augmented by the work of the West Pacific East Asia (WPEA) data improvement project. Additionally, recent data collection efforts in both Indonesia and Vietnam have provided new length-frequency data for inclusion in the recent assessments for both the domestic Indonesia small-scale fishery (Fishery 23) and the domestic Vietnam small-scale fishery in region 2 (Fishery 32).

Size data were missing for the Indonesian-Philippines ex-EEZ purse seine fishery in region 2 (Fishery 24). Based on an investigation of the length frequency data of the other tropical tunas available for this fishery, selectivity was assumed to be shared with the Philippines small-fish fishery in region 2 (Fishery 17) as this fishery had the most similar size composition for the other tropical tuna species.

As in the previous assessments the length frequency samples from the Philippines domestic small fish miscellaneous fishery (Fishery 17) were adjusted to exclude all reported fish lengths greater than 90 cm from the current assessment. These large fish were also excluded from the new length-frequency data for both the domestic Indonesia small-scale fishery in region 2 (Fishery 23) and the domestic Vietnam small-scale fishery in region 2 (Fishery 32). This was done on the basis that it is suspected that the presence of these large fish may be due to mis-reporting of the fishing gear in some of the regional sampling programs.

The Indonesia--Philippines domestic handline fishery in region 2 (Fishery 18) consistently catches the largest individuals in the WCPO. Handline fishing often takes place on mixed--gear trips with other gears such as hook-and-line targeting smaller fish. To avoid \enquote{contaminating} the length-frequency data for this fishery with fish that were mis-reported as being caught using a handline, fish smaller than 70 cm were excluded.

Length data from the Japanese coastal purse-seine and pole-and-line fleets were provided by the National Research Institute of Far Seas Fisheries (NRIFSF). For the equatorial pole-and-line fishery, length data were available from the Japanese distant-water fleet (sourced from NRIFSF) and from the domestic fleets (Solomon Islands and PNG). Since the late 1990s, most of the length data were collected by observers covering the Solomon Islands pole-and-line fleet.

\subsection{Tagging data}
\label{sec:tag_data}

A reasonable amount of tagging data is available for yellowfin tuna, although it is mostly constrained to the tropical region. Information on the yellowfin tag data characteristics and the process of constructing the MFCL tagging file are available in \citet{peatman_analysis_2023-1,teears_cpue_2023}. A summary of the tagging data is in \autoref{fig:tag_release_recapt}, and maps displaying tag displacements are in \autoref{fig:tag_displacement_map}. Data were available from the Regional Tuna Tagging Project (RTTP) during 1989--92 (including affiliated in-country projects in the Solomon Islands, Kiribati, Fiji and the Philippines), historical (1995, 1999-2001) data from the Coral Sea tagging cruises by CSIRO \citep{evans_behaviour_2008}, and the ongoing Pacific Tuna Tagging Programme (PTTP) that began in 2006. Data for the PTTP is included up until the end of 2021, with tag releases included until end of 2020 and recaptures until end of 2021. The 2020 assessment added data from the Japanese Tagging Programme (JPTP) conducted by NRIFSF and the Ajinomoto Co. Inc, over the period 2000--2020, and these data are included in the 2023 assessment. The new tagging data for the 2023 assessment comes primarily from PTTP.

Tags were released using standard tuna tagging equipment and techniques by trained scientists and technicians. Tags have been returned from a range of fisheries, having been recovered onboard or via processing and unloading facilities throughout the Asia-Pacific region.

In this assessment, the numbers of tag releases input to the assessment model were adjusted for a number of sources of tag loss, unusable recaptures due to lack of adequately resolved recapture data, estimates of tag loss (shedding and initial mortality) due to variable skill of taggers (i.e., tagger effects), and estimates of base levels of tag shedding and tag mortality. These adjustments are described in more detail in \citet{peatman_analysis_2023-1}. An additional issue for the yellowfin assessment is that there are tag returns that were released within the WCPO but recaptured to the east of longitude 150\degree{}W, outside the WCPO assessment region. The adjustment or rescaling of releases for recaptures in the EPO preserves the recovery rates of tags from individual tag groups that would otherwise be biased low given that a considerable proportion of recaptures cannot be attributed to a recapture category in the assessment. These procedures were first described in \citet{berger_analysis_2014} and \citet{mckechnie_construction_2016}. For the current assessment, \citet{peatman_analysis_2023-1} and \citet{teears_cpue_2023} describe the approaches to prepare the tagging data. Additionally, the model used to adjust tag releases due to variability in tagger ability or \enquote{tagger effects} has changed from that used in the 2020 assessment. This change was the outcome of an expert workshop to review and recommend the approach for modelling tagger effects and providing the correction factors to adjust the tag release numbers \citep{peatman_analysis_2022-1}. The approach recommend from that workshop was applied to the 2022 WCPO skipjack assessment \citep{castillo_jordan_stock_2022} and is applied to this assessment. The new approach differs from that applied in 2020, in that it reintroduces individual tagging events as a term in the model selection process whilst also keeping cruise leg covariates, whose inclusion were supported for PTTP bigeye releases. It also estimates separate models for central Pacific and western Pacific cruises, given their difference in tagging platforms and associated station and tagger effects, but fits models pooling both yellowfin and bigeye tuna releases, allowing species-specific differences in tagging effects to be accounted for where supported by the data. These changes result in stronger tagger effects being predicted and therefore generally larger adjustment (reductions) to tag releases. This has an important effect of increasing the recapture rates, which has implications for model estimation of fishing mortality and population scale.

After tagged fish are recaptured, there is often a delay before the tag is reported and the data are entered into the tagging databases. If this delay is significant then reported recapture rates for very recent release events will be biased low and will impact estimates of fishing mortality in the terminal time periods of the assessment. For this reason, any release events occurring after the end of 2020 were excluded from the assessment, as noted.

For incorporation into the assessment, tag releases were stratified by release region, time period of release (quarter) and the same size classes used to stratify the length-frequency data.

The likelihood penalties or \enquote{priors} used for the reporting rates of the grouped tag return fisheries has been updated relative to those used in the previous assessment based on the analysis of tag seeding experiments \citep{peatman_analysis_2023-2}. Tag reporting was assumed to be similar between the RTTP and CSTP (which were actually targeted cruises of the RTTP) so reporting rates estimates were shared across these two programs to reduce model dimensionality. For this assessment we have also excluded tag release groups with 5 or less recaptures from the estimation of reporting rates, as we felt there was insufficient information to inform model estimation of the reporting rates. Tag reporting rate groupings are included in \autoref{tab:fisheries_definitions}.
