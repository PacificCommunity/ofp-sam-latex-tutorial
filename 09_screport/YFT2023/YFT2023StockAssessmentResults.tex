% -*- TeX-master: "YFT2023.tex"; eval: (longlines-mode); fill-column: 100000 -*-

\subsection{Stock assessment results}
\label{sec:stock_assess_results}

\subsubsection{Recruitment: diagnostic model}
\label{sec:recruitment_diag}

The estimated recruitment aggregated across all regions (\autoref{fig:recruitment_annual_all}) shows interannual variation, especially in the earlier decades of the assessment period that does not have size data to inform recruitment estimation. The variation in recruitment estimates also fluctuates in the last few years, as these fish are still quite young and have in many cases not yet been observed in the fisheries data. The total recruitment for the final 6 quarters is set to the arithmetic mean recruitment, a model setting that affects only the last two points in \autoref{fig:recruitment_annual_all}. The long-term trend is that the estimated recruitment is somewhat higher in the first decade and last five years, but these are also the periods where recruitment estimates are least reliable.

Overall, region 2 is estimated to have contributed around 40\% of the recruitment to the stock, while regions 1, 4, and 5 are close to 20\% each (\autoref{fig:proportion_rec_region_qutr}). Region 3 is estimated to have practically no recruitment, but regions with essentially zero recruitment have been an unwanted and recurring feature in previous yellowfin assessments \citep{davies_stock_2014,tremblay-boyer_stock_2017,vincent_stock_2020} with the geographic location of missing recruitment varying between the assessments \citep{hamer_report_2023}. The substantial recruitment estimated in region 2 feeds into regions 1, 3, and 5, where region 2 is the largest source of unfished total biomass (\autoref{fig:proportions_biomass_by_source_fmult0}). Region 4, on the other hand, has more equally distributed source regions.

The temporal recruitment trends within individual regions (\autoref{fig:recruitment_annual_regions}) vary more than the sum of all regions (\autoref{fig:spawnpot_rec_biomass_panel}). The estimated recruitment in region 2 has increased steadily since 1990 and currently contributes around 50\% of recruits to the stock. Region 1 has relatively high estimated recruitment between 1980 and 2000, while recruitment in region 4 is estimated very low from 1990 to 2010. Region 5 has a more stable long-term trend in recruitment, and region 3 has effectively zero estimated recruitment.

The estimated relationship between spawning potential and recruitment is shown in \autoref{fig:stock_RR} with the assumed steepness of 0.8. The estimated recruitment variability is considerably higher in the earlier decades of the stock assessment, when size data are limited. The model excludes the early periods prior to 1968 from the estimation of the stock-recruitment curve.

\subsubsection{Biomass: diagnostic model}
\label{sec:biomass_diag}

The estimated total biomass and spawning potential declined steadily from 1960 to 2000, followed by a relatively stable population size since then (\autoref{fig:spawnpot_rec_biomass_panel}). In 1960, the total biomass and spawning potential, averaged across quarters, are estimated to have been 10.6 and 7.5 million tonnes, respectively, and by 2000 they had declined to 4.3 and 2.4 million tonnes. In 2021, the final year of the assessment, the total biomass and spawning potential are estimated at 4.9 and 2.6 million tonnes.

The long-term trends vary between regions (\autoref{fig:spawnpotential_annual_region_diag}), with most regions having a current spawning potential that is close to one third of the 1960 levels. The exception is region 1, which is around two thirds of the the 1960 levels. These trends match the observed long-term CPUE trends in each region (\autoref{fig:cpue_fit}).

The current spawning potential of 2.6 million tonnes is partioned between region 1 = 460 thousand tonnes, region 2 = 370 thousand tonnes, region 3 = 210 thousand tonnes, region 4 = 980 thousand tonnes, and region 5 = 600 thousand tonnes. The combined higher latitude regions 1 and 5 have a current spawning potential of 1,070 thousand tonnes, while the combined equatorial regions 2, 3, and 4 have 1,550 thousand tonnes. Thus, the proportion of the spawning potential in the higher latitude regions is estimated at 41\%, with 59\% in the equatorial regions. These proportions are similar to the proportions of observed CPUE indices in \autoref{sec:fit_cpue}.

Analyses conducted at the 2022 peer review of the 2020 yellowfin assessment showed that the biomass partitioning between regions is effectively determined by the regional scaling of the observed CPUE indices \citep[Figure 6 in][]{punt_independent_2023}.

\subsubsection{Depletion: diagnostic model}
\label{sec:depletion_diag}

The estimated spawning potential depletion \sbsbfo aggregated across all regions shows an initial gradual decline until 1970, followed by a faster decline to about 2005, and is relatively stable after that (\autoref{fig:depletion_annual_region_diag}). The estimated \sbsbfo for all regions combined in the final year of 2021 is 0.43. This pattern varies regionally, as the current level of \sbsbfo is around 0.50 in regions 1, 3, 4, and 5 but a substantially lower level of 0.25 in region 2.

The steepest decline in \sbsbfo occurred from 1990 to 2000, especially in regions 2 and 4, corresponding to higher levels of annual catches in that decade compared to previous decades (\autoref{fig:catch_hist_regional}). The current level of \sbsbfo is similar to the year 2010 in regions 1, 3, 5, while region 4 shows a gradual increase in this recent period. Region 2 shows a substantial recent decline from 0.39 in 2010 down to 0.25 in 2021.

\subsubsection{Fished (SB) versus unfished (SB$_{F=0}$) spawning potential: diagnostic model}
\label{sec:fished_unfished_diag}

To interpret the trends in spawning depletion it is useful to compare the trends in spawning potential, \sb, with the predicted spawning potential that would have occurred in the absence of fishing, \sbfo, also called the unfished biomass (\autoref{fig:fished_unfished_SB}). Unfished biomass is the denominator in the depletion ratio.

The total unfished biomass follows the same decline as the spawning potential during the first two decades of the stock assessment period, indicating that the steep decline in the estimated spawning potential before 1970 can be primarily explained by the estimated recruitment trends rather than fishing. From 1970 onwards, the unfished biomass is estimated to have been relatively stable for the next decades, until a recent increase in the last few years near the end of the assessment period. This matches the estimated recent recruitment increase seen in \autoref{sec:recruitment_diag}.

The individual regions have comparable long-term trends in the estimated unfished biomass, with a noticable hump around 1990 that can be traced to an estimated recruitment pulse a few years earlier. Region 4 has the highest estimated unfished biomass among the regions, while region~1 stands out for having a current unfished biomass level near its historical maximum. Regional differences in the behaviour of the unfished trajectory are an important component in subsequent differences in \sbsbfo.

\subsubsection{Fishing mortality: diagnostic model}
\label{sec:fish_mortality_diag}

The temporal trend in the adult fishing mortality has been a gradual increase until around 2010 and a slight decline since then (\autoref{fig:juv_adult_F}), averaging 0.13 in the last ten years. Juveniles, as defined by the maturity ogive (\autoref{fig:maturity_ogive}), have fishing mortality rates that are generally around two times higher than that of adults, with annual fluctuations. The large difference between juvenile and adult fishing mortality is a change from the previous yellowfin assessment, where juvenile and adult fishing mortality were both around 0.15 for recent years. During the stepwise development of the current diagnostic model, this change occurs when natural mortality is estimated, leading to a different level and shape of the natural mortality curve.

The juvenile fishing mortality is estimated to have increased rapidly in the last few years, from 0.22 in 2015 to 0.46 in 2021. This increase matches the rapidly increasing catches in fishery 23 (\autoref{fig:catch_by_fishery_other}), consisting of miscellaneous gears in Indonesian waters targeting juvenile fish at age 2 and 3 quarters (\autoref{fig:select_other_age}). The annual catches in fishery 23 have increased from 58 thousand tonnes in 2015 to 169 thousand tonnes in 2021, currently more than double that of any other fishery (\autoref{fig:catch_by_fishery_other}).

Regional comparison shows that fishing mortality rates are highest for the youngest ages in region 2 but higher for the older ages in the temperate regions 1 and 5, while regions 3 and 4 have fishing mortality rates that apply both to the younger and older fish in the population (\autoref{fig:age_specific_F}). Decadal comparison shows a recent increase in juvenile fishing mortality rates and that the age distribution in the population has remained relatively stable over time (\autoref{fig:prop_at_age_F_by_decades}).

\subsection{Multi-model inference: sensitivity analyses and structural uncertainty}
\label{sec:multimodel_inference}

\subsubsection{One-off sensitivity analyses}
\label{sec:oneoff_sensitivities}

Comparisons of the spawning depletion and spawning potential trajectories for the diagnostic model and the related one-off sensitivity models are provided in Figures \ref{fig:one_off_sens_tagmix}, \ref{fig:one_off_sens_steepness}, \ref{fig:one_off_sens_sizedatawt}, and \ref{fig:one_off_sens_agedatawt}.

These comparisons show that estimates of both spawning depletion and spawning potential were somewhat sensitive to the choice of tag mixing period, while spawning depletion was also somewhat sensitive to the assumed steepness value, and spawning potential to the assumed size data weighting.

Under the alternative assumed mixing periods, the depletion trajectories followed similar trajectories, diverging from the late 1970s. Results from the 1 quarter mixing scenario indicating a slightly less depleted state than the assumption of a 2 quarter mixing scenario. In terms of spawning potential, that from the 1 quarter mixing scenario was scaled higher across the time series.

With regards steepness, as expected a lower steepness assumption implied a more depleted stock. There was no impact on the estimated spawning potential.

Assumed size data weighting had small impacts on the estimated spawning depletion trajectory, with the divisor of 10 implying a more depleted stock at the end of the time series, and a slightly different trajectory. The assumed size data weighting also scaled the estimated spawning potential.

The age data weighting also had little impact on the estimated spawning depletion trajectory or the estimated spawning potential (\autoref{fig:one_off_sens_agedatawt}).

\subsubsection{Structural uncertainty grid}
\label{sec:structural_uncertainty_analysis}

Results of the structural uncertainty analysis are summarized in box and violin plots of \fref and \sbrsbfo for the different levels of each of the four axes of uncertainty (\autoref{fig:violin_grid_axes}).

The distribution of recruitment across model regions and quarters for all models in the structural uncertainty grid is summarised in \autoref{fig:violin_rec_props_grid}. Time series of spawning depletion (\sbrsbfo) and spawning potential \sb across grid models are shown in \autoref{fig:grid-depletion} and \autoref{fig:grid_SB}. Majuro and Kobe plots showing the estimates of \fref, \sbrsbfo and \sbsbmsy across all models in the grid are presented in \autoref{fig:majuro_kobe}. The averages and quantiles across the 54 models in the grid for all the reference points and other quantities of interest are presented in \autoref{tab:grid_outcomes}. For key management quantities (\sbrsbfo; \fref; \sbrsbmsy) the table alsos includes the additional estimate of estimation uncertainty for management advice.

The general features of the structural uncertainty analyses are as follows:

\begin{itemize}
  \item The grid contains 54 models that display a moderate range of estimates of stock status relative to reference points, and suggest that, overall, the stock is moderately more depleted than estimates from the 2020 assessment (\autoref{tab:grid_outcomes}).
  \item The results of the jittering process to improve the fit of all models in the grid is shown in \autoref{tab:grid_convergence}. The jittering always achieved to improve the objective function value, but the gradient could often get worse after jittering and the Hessian for a given grid model could become non-positive definite.
  \item The most influential axis was steepness, which displayed results consistent with previous structural uncertainty grids. Models with steepness of 0.95 were the more optimistic compared to the steepness of 0.8 assumed in the diagnostic model, while a steepness of 0.65 was the most pessimistic. The lower the steepness the more depleted the stock and the higher the fishing mortality with respect to \fmsy (\autoref{fig:violin_grid_axes}). The assumed steepness level results in a clear clustering of stock status levels on the Kobe plot (\autoref{fig:majuro_kobe}), particularly in relation to estimates of \sbmsy.
  \item Across the tag mixing period axis, results of models with the 2 quarter mixing period implied a slightly greater level of depletion and higher \ffmsy than those with the 1 quarter mixing period.
  \item The estimates of depletion and fishing mortality for a size composition divisor of 10 were more pessimistic than those for 20 and 40, which were relatively consistent with one another. With an assumed size composition divisor of 10 there was a slight skew in the distribution toward more optimistic stock status.
  \item The conditional age-at-length data weighting axis had limited impact on management quantities, with all levels showing similar ranges of depletion and \ffmsy.
  \item Spawning depletion was generally low in the initial time period and started to increase in the early 1970s. Depletion stabilised in the mid-2000s through to the most recent period.
  \item Spawning depletion estimates in region 2 are approaching the limit reference point \lrp in the most recent model years for some of the grid models (\autoref{fig:grid-depletion}). While not to the same extent, depletion in the remainder of the tropical region (regions 3 and 4) have shown notable declines, but some recovery in the most recent period. Declining trends in the temperate regions are less severe and have also levelled out (\autoref{fig:grid-depletion}).
  \item Similar patterns are seen in spawning potential, with declines most notable in the tropical region, as well as in region 5 (\autoref{fig:grid_SB}).
  \item Recruitment is predicted to be highest in region 2, with moderate recruitment in regions 1, 4 and 5, and lower recruitment in region 3. Across the scenarios, region 3 recruitment estimates are influenced by the tag mixing assumption, with zero recruitment in this region under all 2 quarter mixing scenarios but recruitments present where 1 quarter tag mixing is assumed. There is no consistent pattern in quarterly recruitment between regions (\autoref{fig:violin_rec_props_grid}).
  \item None of the models in the structural uncertainty grid had an overall spawning potential depletion below the LRP (\lrp); median \sbrsbfo was 0.47 (80 percentile range: 0.42 to 0.52) (\autoref{tab:grid_outcomes}).
  \item All models in the structural uncertainty grid showed exploitation to be below \fmsy. Median \fref was 0.50 (80 percentile range 0.42 to 0.61).
\end{itemize}

\subsubsection{Integration of estimation and model uncertainty for key management quantities}
\label{sec:estimation_uncertainty_analysis}

Estimation uncertainty across the grid of 54 models was calculated for the key management quantities \sbrsbfo, \fref and \sbrsbmsy (\autoref{tab:grid_outcomes}). Distributions of the resulting quantities broken down by element for each of the four grid axes are presented in \autoref{fig:hist_dep} for \sbrsbfo, \autoref{fig:hist_fref} for \fref and \autoref{fig:hist_sbsbmsy} for \sbrsbmsy.

Presenting the estimates arising from these two approaches to incorporating uncertainty, results from the uncertainty grid of models and incorporating estimation uncertainty, allows the impact of the additional estimation uncertainty to be examined.

The median values for \sbrsbfo from the grid and after incorporating estimation uncertainty are identical. The tails of the distribution are also the same, after rounding to two decimal places.

For MSY-related quantities, incorporation of estimation uncertainty had slightly larger impacts than the changes seen when incorporating estimation uncertainty into \sbrsbfo. Median \fref is the same but median \sbrsbmsy is slightly lower when incorporating estimation uncertainty. d the 80 percentile ranges of \fref and \sbrsbmsy become somewhat wider, with some values below the 10th percentile of \sbrsbmsy falling below 1. These values are influenced by the levels of mixing and assumed steepness with mix 2 and lower steepness assumptions leading to higher estimates of \fref (\autoref{fig:hist_fref}) and lower estimates of \sbrsbmsy (\autoref{fig:hist_sbsbmsy}).

It is recommended that management advice is based on the estimated management quantities including both the uncertainty grid and estimation uncertainty. The values of \sbrsbfo are all above the LRP (20\% \sbfo SBF=0), and the values of \fref are all below 1.

\subsubsection{Analyses of stock status}
\label{sec:stock_status_analysis}

There are several ancillary analyses related to stock status that are typically undertaken on the diagnostic model (e.g., dynamic Majuro and Kobe analyses, fisheries impact analyses etc.).

We do not present the results of all analyses for all models in the stock assessment paper. In this section, we rely largely on the dynamic spawning depletions and spawning potential plots for the models in the structural uncertainty grid (Figures \ref{fig:grid-depletion} and \ref{fig:grid_SB}). We also refer to the fished and unfished spawning potential trajectories for the diagnostic model discussed previously (\autoref{fig:fished_unfished_SB}) and the dynamic Majuro and Kobe plots (\autoref{fig:dynamic_majuro_kobe}).

\textbf{Dynamic Majuro and Kobe plots and comparisons with Limit Reference Points:} The section summarizing the structural uncertainty grid (\autoref{sec:structural_uncertainty_analysis}) presents terminal estimates of stock status in the form of Majuro plots. Further analyses can estimate the time series of stock status in the form of Majuro and Kobe plots, the methods of which are presented in \autoref{sec:method_kobe}. The dynamic Majuro and Kobe plots for the diagnostic model are presented in \autoref{fig:dynamic_majuro_kobe}.

Both the dynamic Majuro and Kobe plots show the steady increase in depletion of the stock since the 1950s, with an increase in fishing mortality from the 1960s. The dynamic Majuro plot indicates that while depletion stabilised toward the end of the assessment time period, fishing mortality tended to increase. However, the terminal spawning potential is well above both \sbmsy and \lrp, and the fishing mortality is well below \fmsy (\autoref{fig:dynamic_majuro_kobe}).

\textbf{Fishing impact:} In addition to the above analysis, it is possible to attribute the fishery impact with respect to depletion levels to specific fishery components (i.e., grouped by gear type), to estimate which types of fishing activity have the most impact on the spawning potential (\autoref{fig:fishery_impact}). Fishing impacts were estimated to be very minor in all regions before about 1970, resulting primarily due to longline and pole and line fisheries. The impact of these gears has increased slightly over the time series. In the early 1970s, catch information from the miscellaneous fisheries leads to an increase in impact, with the onset of notable impacts due to purse seine fishing from the 1980s. Examining the overall impact, the miscellaneous and purse seine fisheries (associated and unassociated sets) have the major impact, with that of the miscellaneous fisheries increasing in the most recent period, and the impact of purse seine being equally split between associated and unassociated sets towards the end of the time series.

The greatest fishing impact is in region 2, where the miscellaneous fishery in this region has the largest proportional impact. The impact of the miscellaneous gears is also seen to a lesser extent in other regions, due to movement. Impact in the other tropical regions (regions 3 and 4) are primarily due to purse seine fishing, with a similar pattern seen in region 5.

\textbf{Yield analysis:} The yield analysis conducted in this assessment incorporates the spawner recruitment relationship (\autoref{fig:stock_RR}) into the equilibrium biomass and yield computations. Importantly, in the diagnostic model, the steepness of the SRR was fixed at 0.8 so only the scaling parameter was estimated. Other models in the one-off sensitivity analyses and structural uncertainty analyses assumed steepness values of 0.65 and 0.95.

The yield distributions under different values of fishing effort relative to the current effort are shown in \autoref{fig:yield} for select models representing different axes of the structural uncertainty grid (specifically, different levels of steepness). For the diagnostic model, it is estimated that MSY would be achieved by increasing fishing mortality by a factor of 1.65, although the resulting increase in yield would be relatively small (10\%). The different example yield curves under the alternative steepness assumptions display a similar pattern over the scale of fishing mortality although the absolute value of the yield curve and behaviour of the descending limb differs significantly.

The yield analysis also enables an assessment of the MSY level that would be theoretically achievable under the different patterns of age-specific fishing mortality observed through the history of the fishery. We present a plot for the diagnostic case model in \autoref{fig:dynamic_msy}. Prior to 1970, the WCPO yellowfin fishery was almost exclusively conducted using LL gear, with a low exploitation of small yellowfin. Fisheries other than longline were known to operate in the region before 1970, but no catch estimates are available. The associated age-specific selectivity pattern resulted in a much higher MSY in the early period compared to the recent estimates. A pronounced decline occurred after the expansion of the juvenile fisheries in region 2 and, soon after, the rapid expansion of the PS fishery which shifted the age composition of the catch toward younger fish.
