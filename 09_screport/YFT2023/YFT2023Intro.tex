% -*- TeX-master: "YFT2023.tex"; eval: (longlines-mode); fill-column: 100000 -*-

\clearpage

\section{Introduction}
\label{sec:introduction}

This paper presents the 2023 stock assessment of yellowfin tuna (\emph{Thunnus albacares}; YFT) in the western and central Pacific Ocean (WCPO; west of 150\degree W). Assessment of WCPO yellowfin tuna has been conducted regularly since the late 1990s \citep{hampton_stock_2003,hampton_stock_2005-1,langley_stock_2011,davies_stock_2014,tremblay-boyer_stock_2017,vincent_stock_2020}. As in previous assessments, the objectives of the 2023 yellowfin tuna assessment are to estimate population level parameters which indicate the stock status and impacts of fishing, such as time series of recruitment, biomass, biomass depletion and fishing mortality. We summarize the stock status in terms of reference points adopted by the Western and Central Pacific Fisheries Commission (WCPFC). The methodology used for the assessment is based on the general approach of integrated modeling \citep{fournier_general-theory_1982}, which is carried out using the stock assessment framework MULTIFAN-CL\footnote{http://www.multifan-cl.org} (MFCL version number 2.2.x.0; \citealp{fournier_multifan-cl_1998,hampton_spatially-disaggregated_2001,kleiber_multifan-cl_2019}). MFCL implements a size-based, age- and spatially-structured population model. Model parameters are estimated by maximizing an objective function, consisting of both likelihood (data) and \enquote{prior}\footnote{Note that any mention of a \enquote{prior} in this report does not refer to a prior in the Bayesian sense, though the effect on the parameter estimate is similar, but rather a penalty placed on the likelihood such that the estimated parameter does not deviate too much from the specified \enquote{prior} value. The magnitude of the deviation from the \enquote{prior} is dependent on the information content of the data and the strength of the likelihood penalty applied.} information components (penalties).

Each new assessment of a WCPO tuna stock typically involves updates to fishery catch, effort, and size composition data, updates to tag-recapture data when tagging data is used, implementation of new features in the MFCL modeling software, changes to preparatory data analysis, such as CPUE standardisations, and consideration of new information on biology, population structure and potentially other population parameters. These changes are an important part of efforts to continually improve the modeling procedures and more accurately estimate fishing impacts, biological and population processes and quantities used for management advice. Advice from the Scientific Committee (SC) on previous assessments, and the annual SPC pre-assessment workshops (PAW) \citep{hamer_report_2023} guides this ongoing process. Furthermore, due to changes in assessment staff, new assessments often involves staff that did not participate in the previous versions and this may also influence differences in how assessment are conducted. Changes to aspects of an assessment can result in changes to the estimated status of the stock and fishing impacts, and resultant management advice. It is important to recognize that each new assessment represents a new estimation of the historical population dynamics and recent stock status, and each new assessment team strive to provide the best possible assessment with the time and data available.

The assessment uses an `uncertainty grid' of models as the basis for management advice. The uncertainty grid is a group of models that are run to explore the interactions among selected \enquote{axes} of uncertainty that relate to biological assumptions, data inputs and data treatment. The axes are generally selected from one-off sensitivity models of a diagnostic (or base case) model to indicate uncertainties that have notable effects on the estimates of key model parameters and management quantities. The variation in estimates of the key management quantities across the uncertainty grid represents the uncertainty in stock status and should be considered carefully by managers. This structural or `model' uncertainty is usually more important than the uncertainty in the estimation of parameters from individual models, referred to as `estimation uncertainty'. However, both are taken into account when documenting the uncertainty in the key management quantities provided by this assessment.

The 2023 yellowfin tuna assessment occurs after the 2022 peer review of the WCPO 2020 yellowfin tuna assessment \citep{punt_independent_2023}. The peer review outcomes have implications for the current assessment and these are noted where relevant. Notable new features of the 2023 assessment are summarised below and this assessment report should be read in conjunction with several supporting papers, specifically the paper on CPUE analyses and other data inputs \citep{teears_cpue_2023}, the paper on size composition data preparations and weighting \citep{peatman_analysis_2023}, the papers on tag reporting rates and tagger effects estimations \citep{peatman_analysis_2023-2,peatman_analysis_2023-1}, the paper on improved conversion factors and data on fish weights and lengths \citep{macdonald_project_2023-1}, the paper on developments in the MFCL software \citep{davies_developments_2023} and the paper on conceptual models of yellowfin  and bigeye population structure \citep{hamer_review_2023}. Finally, the planning for this assessment was informed by the discussions at the 2023 SPC PAW \citep{hamer_report_2023}.

Significant changes and improvements to the analysis used in this assessment include the following, which are discussed in more detail in relevant sections of this report.

\begin{itemize}
  \item Conversion from a catch-errors to a catch-conditioned approach, and the inclusion of a likelihood component for the CPUE from the index fisheries (peer review supported this).
  \item Adoption of a simpler spatial structure (5 model regions). Detailed review of information and development of conceptual models for spatial structure including both size composition analysis (regression trees) and CPUE time series analysis (peer review recommendation, supported by PAW).
  \item Change from using VAST to sdmTMB to standardise the input CPUE series and increased the spatial resolution of the mesh configuration. Various alternative CPUE model structures and analyses explored resulting in the inclusion of additional covariates in the CPUE model (peer review recommendation).
  \item Different CPUE variances were used for the CPUE associated with each index fishery, using new approaches to estimate these variances. Modifications were required to MFCL to enable the index fisheries to have separate CPUE variances while maintaining shared selectivity  (peer review recommendation).
  \item Internal estimation of natural mortality and application of the Lorenzen form of natural mortality (recommendation of 2023 CAPAM Tuna Good Practices Workshop), also an MFCL modification to allow input of Lorenzen starting parameter values with improved parameter scaling.
  \item Additional procedures implemented for achieving more reliable model convergence, including jittering and checking positive definite Hessian status for all grid models (improvements to convergence criteria requested by SC18, recommendation by peer review to provide Hessian diagnostics).
  \item Integration of estimation uncertainty with model-based uncertainty across the grid (SC18 request for inclusion of estimation uncertainty). An MFCL development to enable calculation of variances only for the key derived quantities required for the uncertainty grid was implemented, reducing the computational load to estimate uncertainty for management quantities.
  \item Use of MFCL tail compression feature, applied only to zero values (and could apply a further tail compression proportion of 0.001 in future).
  \item Improved size composition data filtering approaches to reduce influence of low/unrepresentative sampling. Also explored alternatives for specifying input samples sizes, such as numbers of sets (but ran out of time to fully explore a range of filtering options). Applied a minimum input sample size of 50 for size composition data in MFCL (peer review recommendation to reduce unrepresentative size composition data).
  \item Continued use of conditional age-at-length data and internal estimation of growth, but downweighting this data in the likelihood as an initial step in data weighting (peer review recommendations).
  \item Reduced the number of fisheries with a non-decreasing selectivity constraint to just the index fisheries.
  \item Ensured that tag reporting rate groups are not estimated for groups with zero tag recoveries, and extending this to tag reporting rate groups with fewer than 6 tag recoveries.
  \item Adoption of revised tagger effect modelling framework (recommended by expert workshop) with separate treatment of PTTP Central Pacific tag releases; use of multi-species models; model selection based on predictive accuracy; and, reverted to assumptions similar to those used in 2017.
  \item Initial explorations of the use of Dirichlet multinomial for self-scaling size composition data weighting (peer review supported this), and modification (reduction) to size composition data weighting divisors in the grid as a result.
  \item Qualitative analysis of tag recapture data to inform tag mixing assumptions (PAW recommendation, and peer review).
\end{itemize}

\section{Background}
\label{sec:background}

\subsection{Stock structure}
\label{sec:stock_structure}

Yellowfin tuna is distributed across the Pacific, Indian and Atlantic Oceans in a continuous band between approximately 45\degree north and 45\degree south of the equator \citep{grewe_evidence_2015,moore_defining_2020}. Genetic studies suggest that populations in the three major oceans are largely separate \citep{pecoraro_population_2018,moore_defining_2020}, although connectivity between yellowfin spawning areas in the Indian Ocean and populations in the Atlantic Ocean near south Africa has been detected \citep{mullins_genomic_2018}. \citet{grewe_evidence_2015} showed strong genetic differences between samples from Baja California in the eastern Pacific, Tokelau in the central Pacific and the Coral Sea in the western Pacific. Similarly, \citet{pecoraro_population_2018} found genetic differences between populations in the far eastern and western Pacific Ocean. Evidence of finer scale genetic structure of yellowfin in the western and central Pacific Ocean is less clear and varies between different studies \citep{appleyard_population_2001,aguila_distinct_2015,pecoraro_population_2018,anderson_close_2019,evans_connectivity_2019}. An earlier genetic study by \citet{ward_allozyme_1994} proposed the existence of eastern and western Pacific sub-populations separated at around 150\degree~W. Observations of the distribution of yellowfin tuna larvae indicate that spawning occurs broadly throughout the central and western tropical Pacific, with spawning all year in the equatorial region, and seasonally in warmer months to the north and south \citep{nishikawa_average_1985,ijima_tuna_2023}. Studies using otolith chemistry have suggested that populations at sub-regional scales may be sourced predominantly from local spawning \citep{wells_nursery_2012,rooker_natal_2016,proctor_population_2019}. The most recent otolith chemistry study provided good evidence that majority of small (~30-40 cm) juvenile yellowfin captured around Japan originated from very small juveniles ($<$ 10 cm) sampled further south in the western Pacific tropical region \citep{satoh_connectivity_2023}. This suggests dispersal/movement from the equatorial western Pacific spawning areas via the western boundary current (Kuroshio) is important for juvenile recruitment around Japan.

The results of genetic studies are broadly consistent with tag/recapture data in suggesting that mixing between the far western and far eastern Pacific Ocean regions is limited \citep{moore_defining_2020,hamer_review_2023}. The extensive tag/recapture data available since 1989 shows that longitudinal movements among the equatorial regions of the central and western Pacific can be extensive but latitudinal movements to and from the tropical/sub-tropical latitudes may be less so (\cref{fig:tag_displacement_map}). The longitudinal movements and continuous distribution across the Pacific suggests an isolation by distance mechanism is responsible for the genetic differences observed between the west and east Pacific populations. Despite the significant tagging and recent genetic studies, there remains considerable uncertainty on sub-regional population structure in the WCPO, in particular the spatial connectivity between spawning areas, early life stages, and recruitment to size classes vulnerable to fishing in different regions. This is an important area for further research. A review of the status of knowledge of yellowfin tuna population structure is available in \citet{moore_defining_2020}, and concludes that the weight of evidence from both genetic and non-genetic studies supports the presence of discrete stocks of yellowfin tuna in the EPO and WCPO, as well as the potential for finer-scale spatial structuring within each of these regions. For this assessment, the stock within the domain of the model area (essentially the WCPO, west of 150\degree W) has been considered as a discrete stock that exhibits the same biological traits \citep{langley_stock_2011,davies_stock_2014,tremblay-boyer_stock_2017}.

Over time, the spatial complexity of the modeling of the yellowfin stock in the WCPO has increased. In the 2011 assessment the model domain was divided into 6 regions. After a review of the bigeye assessment in 2012 \citep{ianelli_independent_2012}, a 9 region model (\cref{fig:map_5regions_9regions}) was implemented in 2014 for both yellowfin and bigeye \citep{davies_stock_2014,harley_stock_2014,mckechnie_basis_2014}, with the northern boundary of regions 3 and 4 set at 20\degree N. In the 2017 yellowfin assessment \citep{tremblay-boyer_stock_2017} an additional option was included that involved moving the northern boundary of regions 3 and 4 to 10\degree N to better reflect the purse seine fishery spatial structure and the assumption of low movement rates between the equatorial and sub-tropical northern regions (\cref{fig:tag_displacement_map}). Based on the comparisons between the 10 and 20\degree N options in the 2017 assessment, and the same comparisons for the concurrent bigeye assessment \citep{vincent_incorporation_2018}, the 2020 PAW recommended only using the 10\degree N option in the 2020 yellowfin assessment. The 9 region spatial structure was viewed as a compromise between the limited knowledge of sub-regional population structure, fishery/fleet spatial structures and the locations of major localized tag release events (i.e. regions 4, 8 and 9).

The paper by \citet{hamer_review_2023} considers published information on genetic and non-genetics indicators of population structure (mostly covered in the review by \citet{moore_defining_2020}), larval distribution patterns, and also includes analyses of spatial heterogeneities in size composition data and CPUE time series for the Pacific longline fisheries. That paper suggests that while yellowfin tuna are likely one genetic stock in the Western and Central Pacific Ocean (WCPO), there is indication of substructure with the WCPO assessment region. The paper notes that the tropical region has several features that warrant it being considered as a separate spatial strata from the northern and southern sub-tropical/temperate regions. Likewise the paper suggests that the Indonesia/Philippines/Vietnam/South China Sea region warrants being considered a separate spatial strata. There was also support for a separate region/fisheries around Hawaii based on size composition data, localised spawning and otolith chemistry studies. The review did not conclude that the previous 9 region model structure was inappropriate but did suggest alternative simplified spatial stratifications that could be considered for yellowfin tuna assessment in the WCPO and this is discussed further in \autoref{sec:spatial_stratification}. Continuing to improve understanding of spatial population structure and processes for yellowfin in the WCPO, and more broadly in the Pacific, remains an important area of research.

The 9 region spatial structure (\autoref{fig:map_5regions_9regions}) applied in the 2020 assessment was the basis for the current assessment. However, based on the review and analyses in \citet{hamer_review_2023} and the recommendation from the peer review to explore plausible simpler spatial structures \citep{punt_independent_2023}, we considered an alternative simplified 5 region structure. Ultimately, the 5 region structure was adopted for this assessment due to better model performance and convergence properties (see \autoref{sec:stepwise}), including the requirement to achieve a positive definite Hessian. A thorough exploration of alternative spatial structures besides the 5 region structure was not possible in the time available, but is still recommended as future work.

\subsection{Biological characteristics}
\label{sec:biology}

Yellowfin tuna can reach a maximum fork length (FL) of around 180 cm and live for up to 15 years, although most fish aged to date have been less than 10 years old \citep{itano_reproductive_2000,farley_age_2020} and the maximum validated age is 13 years \citep{andrews_final_2022}. Growth of the juvenile stage is particularly fast and they can reach a fork length of around 20-30 cm by three months of age and approximately 50 cm by 1 year \citep{farley_age_2020}. Length at 50\% maturity in the WCPO is at around 100-110 cm \citep{itano_reproductive_2000} which equates to around 2 years of age. In this assessment, for the purpose of computing the spawning biomass, we assume a fixed maturity schedule consistent with the observations of \citet{itano_reproductive_2000} (see \citealp{vincent_background_2020} for details). Yellowfin tuna are thought to spawn opportunistically throughout the Pacific in waters warmer than 24\degree C \citep{itano_reproductive_2000,reglero_worldwide_2014}. Larval stages are found widely in surface waters throughout the central and western Pacific \citep{nishikawa_average_1985,servidad-bacordo_composition_2012,ijima_tuna_2023} and at least some spawning appears to occur year-round in the WCPO. However, understanding of spatio-temporal variation in spawning fraction is limited. Important areas for spawning are thought to occur in the Banda Sea in Indonesia, the north-western Coral Sea, the eastern and southern Philippines, northeast of Solomon Islands, and around Fiji \citep{mcpherson_reproductive_1991,gunn_origin_2002,servidad-bacordo_composition_2012,ijima_tuna_2023}. Juvenile yellowfin (several months of age) are prevalent in commercial fisheries in the Philippines and eastern Indonesia \citep{hare_compendium_2023,williams_overview_2023}, suggesting this region is important for the juvenile stages, but they are also found more widely in the equatorial Pacific.

Growth parameters can be highly influential in the estimates of management parameters by stock assessment models, and the most recent stock assessments of yellowfin in the WCPO identified growth as an important uncertainty requiring further research \citep{mckechnie_stock_2017,vincent_incorporation_2018, vincent_stock_2020}. Considerable work was carried our on developing external growth curves from otoliths and otolith-tag integrated growth models for the previous assessment \citep{eveson_integrated_2020,farley_age_2020}. Those studies resulted in different estimates of growth rate parameters to those previously estimated by the MFCL model based on length composition modal progression. Annual aging of yellowfin using otoliths has recently been validated using the bomb radiocarbon method \citep{andrews_final_2022}, but there has been no further otolith aging of yellowfin since the 2020 assessment. The previous assessment explored alternative growth options including external fixed growth curves from both otolith aging an internal modal length analyses (applied as external fixed curves), as well as a fully internal growth estimation by inputting the otolith data (with associated length data) as conditional-age-at-length data. The latter was used as the approach for the diagnostic model, and has since been recommended as the most suitable approach for growth estimation by the yellowfin assessment peer review \citep{punt_independent_2023}. Understanding of spatio-temporal variation in growth is limited and is insufficient to consider such effects in the current assessment, nor is it feasible to implement spatially varying growth using the MFCL model framework.

Natural mortality ($M$) rate of yellowfin tuna varies with size/age \citep{hampton_natural_2000}. Mortality is highest for the smaller juveniles and estimated to be lowest for the pre-adult stage (50-80 cm FL) 0.6-0.8 $yr^{-1}$ \citep{hampton_natural_2000}. After reaching maturity it is thought that mortality increases with age, particularly in females. Sex ratios of yellowfin tuna are commonly observed to be biased toward males at larger sizes, and it is thought that this may relate to the higher mortality rates of mature females due to the physiological stresses related to spawning or a combination of this and different growth rates between males and females at older ages \citep{schaefer_mb_synopsis_1963,hampton_natural_2000,fonteneau_estimated_2002,sun_reproductive_2006,zhu_reproductive_2008}. For the purpose of computing the spawning potential and mortality schedules, data on sex ratio at length is important. For the previous assessment data on sex ratios collected by observers in the WCPO was incorporated into the estimation of spawning potential and $M$ at age \citep{vincent_background_2020}. We utilise the same data source in this assessment. The 2020 yellowfin assessment conducted a life-history based meta-analysis of natural mortality for yellowfin \citep{vincent_background_2020} indicating an envelope of potential quarterly $M$ rates of lower 95\% confidence interval (0.1100), mean (0.1298) and upper 95\% confidence interval (0.1495). More recently natural mortality of yellowfin has been reviewed by \citet{hoyle_approaches_2023}, and this review is taken into account when considering options for natural mortality in the current assessment.

\subsection{Fisheries}
\label{sec:fisheries}

Yellowfin tuna is an important component of tuna fisheries throughout the WCPO. They are harvested with a wide variety of gear types, from small-scale artisanal fisheries in Pacific Island and southeast Asian waters to large, distant-water longliners and purse seiners that operate widely in equatorial and tropical waters \citep{williams_overview_2023}. Purse seiners catch a wide size range of yellowfin tuna, however, smaller yellowfin often dominate catches associated with FADs (fish aggregation devices), whereas the longline fishery takes mostly larger adult fish \citep{vidal_developing_2020, williams_overview_2023}.

The annual yellowfin tuna catch in the WCPO increased from 100,000 mt in 1970 to between 700,000 and 750,000 mt in recent years, mainly due to increased catches in the purse seine fishery, \citep{hare_compendium_2023,williams_overview_2023}. The 2022 catch was approximately 720,000 mt, and the 2021 (last of year of this assessment) catch was slightly higher at approximately 730,000 mt (\cref{fig:catch_hist_full}). Purse seiners harvest the majority of the yellowfin tuna catch (around 50--55\% since 2018), while the longline fleet accounted for around 10-15\% of the catch in recent years, primarily in the equatorial regions (\cref{fig:catch_hist_regional}, \citealp{williams_overview_2023}). The remainder of the catch is dominated by the domestic fisheries of the Philippines and Indonesia, principally catching smaller individuals using a variety of small-scale gear types (e.g. pole-and-line, ringnet, gillnet, handline and seine net). Small to medium sized purse seiners based in those countries also catch fish of sizes more typical of the purse seine fisheries elsewhere.

Yellowfin tuna typically represent 15--20\% of the overall purse-seine catch in recent years and may contribute higher percentages of the catch in individual sets. Yellowfin tuna are often directly targeted by purse seiners, especially within unassociated schools (free schools) that are comprised of larger yellowfin compared to those associated with FADs (associated sets). Unassociated sets account for the majority of purse seine sets, however, many of these sets fail (skunks) and when considering successful sets only, the numbers of associated and unassociated sets are similar \citep{hare_western_2022}.

Since 2010, annual catches of yellowfin tuna by longline vessels in the WCPO have varied between approximately 74,000 to 105,000 mt \citep{williams_overview_2023}. The highest longline catch recorded was around 125,000 mt in 1980 (\cref{fig:catch_hist_full}). Annual catches from the domestic fisheries of the Philippines and eastern Indonesia area are highly uncertain, particularly prior to 1990. Recent estimates for pole and line and other gears have reached approximately 220,000--260,000 mt in the last 5 years years, and for purse seine 350,000--400,000 mt (\cref{fig:catch_hist_full}).

\cref{fig:catch_map} shows the spatial distribution of yellowfin tuna catch in the WCPO for the past 10 years. Most of the catch is taken in western equatorial areas, with less catch by both purse-seine and longline toward the east. The east-west distribution of catch is strongly influenced by ENSO events, with larger catches taken east of 160 \degree E during El Ni\~no episodes. Catches from outside the equatorial region are relatively minor (5\%) and are dominated by longline catches south of the equator and purse-seine and pole-and-line catches in the north-western area of the WCPO (\autoref{fig:catch_hist_regional} and \autoref{fig:catch_map}).

Improved catch statistics in recent years for the Indonesian, Philippines, and Vietnamese fleets have resulted from collaborative work between the fisheries agencies of these countries and the SPC, WCPFC, and CSIRO. In some instances data are available at the individual fisheries level (e.g., longline or large-fish handline), but often statistics are aggregated across a variety of gears that typically catch small yellowfin tuna, e.g., ring-net, handline, and troll. Data for these fisheries have been included in this assessment.

\subsection{Key changes from the last assessment}
\label{sec:key_changes}

\subsubsection{Catch conditioned approach}
\label{sec:catch_cond}

In previous MULTIFAN-CL assessments of yellowfin tuna, catch was predicted by the model (termed `catch-errors' model) with observation error allowed, and the standard deviation of the log-catch deviates assumed to be very small (equivalent to a CV of 0.002). This produced very accurate predictions of observed catches and therefore only a small contribution of the catch to the overall objective function. However, the cost of treating the catch in this way was that effort deviation coefficients had to be estimated as model parameters for each catch observation. Additionally, catchability deviation parameters were required for catch-effort observations for fisheries for which time-series changes in catchability were allowed. While these parameters were constrained by prior distributions and estimation was feasible, it resulted in very large numbers of parameters needing to be estimated by the function minimiser and many of these were effort deviation coefficients and parameters relating to catchability.

In an effort to reduce complexity and parameterisation this assessment makes use of a relatively new feature of MULTIFAN-CL first applied to the 2022 skipjack tuna assessment in which catch is assumed to have no error, i.e., the model is `catch-conditioned' \citep{davies_developments_2022}. This makes it possible to solve the catch equation for fishing mortality exactly, using a Newton-Raphson sub-iterative procedure. The main benefit of this approach is that effort deviation coefficients and catchability-related parameters do not require estimation as model parameters. Effort data for extraction fisheries is not required at all but can be used if available to estimate catchability through regressions of fishing mortality and effort, and this is important for making stock projections based on future effort scenarios. The reduction in parameters has enabled more rapid model convergence and Hessian matrix computation. The only cost of this approach is that missing catches, which could be accommodated in the catch-errors version if there was an accompanying effort observation, are no longer straight forward to account for. However, this is not an impediment for the key WCPO tuna assessments. The catch conditioned approach allows (but does not require) the specification of index fisheries to provide indices of relative abundance, these are discussed in \autoref{sec:fisheries_definitions}. In the stepwise model development runs conducted for this assessment, the transition from a `catch-errors' to a `catch-conditioning' model, without implementation of the survey fisheries, did not result in any appreciable change in the estimated quantities of relevance to management advice.
