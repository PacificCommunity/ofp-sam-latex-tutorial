\documentclass[11pt]{article}
\usepackage[a4paper,margin=2.54cm]{geometry}
\usepackage{parskip}
\begin{document}

OFP provides scientific advice on the status of stocks, management options and impacts of fisheries and the environment---including climate change---on both target and non-target species and the pelagic ecosystem; conducts research on the biology and ecology of tuna and tuna-like species; provides technical support in the collection and management of data from fisheries; and provides capacity building opportunities to members across these fields of work.

OFP delivers its work to members both at the national level and through the various regional and sub-regional fisheries organisations of which they are members, including the Western and Central Pacific Fisheries Commission (WCPFC), the Pacific Islands Forum Fisheries Agency (FFA), the Parties to the Nauru Agreement (PNA) and the South Pacific Group (SPG). In doing so, there is extensive collaboration with the Secretariats of these organisations, as well as with government fishery agencies, many NGOs and universities.

Organisationally, the work of the Programme falls into three sections---the Fisheries \& Ecosystems Monitoring \& Assessment (FEMA) section, the Data Management (DM) section and the Stock Assessment \& Modelling (SAM) section---however, there is considerable integration across these areas to provide comprehensive services and support in oceanic fisheries to members.

Providing scientific advice to help maintain healthy oceanic resources and ecosystems is a key OFP role, to ensure that both short-term and long-term options are informed by the best available scientific information. OFP works collaboratively to identify emerging issues and deliver scientific advice to improve understanding and help develop effective mitigation strategies. OFP also pursues work that will support sustainable management of tuna fisheries to help ensure longer-term benefits can be maintained. In turn, progressing work that will contribute to a better understanding of food security options, such as estimation of potential bycatch levels, further research into non-target species biology, and data reporting tools (e.g. TAILS), aims to help provide scientific information to support decision making. Regarding climate change induced impacts on regional oceanic resources, OFP, with the assistance of regional partners, has significantly enhanced its capacity to support national and regional decision making by providing the latest information and advice. 

\end{document}
