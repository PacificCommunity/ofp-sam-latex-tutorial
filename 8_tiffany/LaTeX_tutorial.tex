
\documentclass[xcolor=dvipsnames]{beamer}

\usetheme{Singapore}
%\setbeamertemplate{navigation symbols}{}

\usepackage{color, graphicx, tikz, verbatim} % Allows including images
\usepackage{booktabs} % Allows the use of \toprule, 

\usepackage [english]{babel}
\usepackage [autostyle, english = american]{csquotes}
\MakeOuterQuote{"}

%\usecolortheme[RGB={194,91,91}]{structure}


\title{\Huge{\textbf{\LaTeX\\ Document Preparation}}}
\subtitle{}
\author{\textbf{\large{Tiffany Cunningham}}}
\institute{\normalsize{MA Division of Marine Fisheries}}
\date{\normalsize{1 March 2018}}

\begin{document}

\begin{frame}
\titlepage
\end{frame}

\section{Intro}
\subsection{Intro}

\begin{frame}{\LARGE{Objectives}}
\begin{itemize}
\large{
\item Basics of document preparation	
\begin{itemize}
\item Create a document!
\item Explore document layout
\item Create an equation
\item Insert a figure \& a table
\item BibTex and references
\item Explore formatting and basic markup tools
\end{itemize}
\item Merging LaTeX with R
\begin{itemize}
	\item Basic example
	\item Figure and table control
	\item Code chunk options
\end{itemize}
\item Beamer for presentations}
\begin{center}

\end{center}
\end{itemize}
\end{frame}

\begin{frame}{\LARGE{What is \LaTeX?}\\ \normalsize{\textit{LAH-TEK} or \textit{LAY-TEK}}}
A typesetting and cross-referencing program for document preparation, presentations, and complex graphics.\\[.5cm]

\textbf{Typesetting:} the art and process of the design and printing of text. Typesetting consists largely of the following:
\begin{itemize}
	\item fitting text and illustrations to a desired page extent
	\item place text and illustrations effectively and appropriately
	\item ensure the layout is uniform and predictable 
	\item produce a final document ready for print.
\end{itemize}
To achieve these objectives some considerations include (but are not limited to): font selection; paragraph and character styles; and spacing between characters, lines, and objects. 
\end{frame}


\begin{frame}{\LARGE{History of \LaTeX}}
\begin{columns}
\begin{column}{.25\textwidth}
	\includegraphics[width=2.5cm, height=3cm]{knuth.jpg}\\
	\includegraphics[width=2.5cm, height=3cm]{lamport.jpg}
\end{column}
\begin{column}{.8\textwidth}
	TeX system developed by Donald Knuth (late 1970s).\\
	TeX came from the capital letters $\tau$, $\epsilon$, and $\chi$; tex is derived from Greek meaning skill, art, technique.\\[2cm]
	LaTeX, developed by Leslie Lamport in the 1980s, incorporates TeX with built in macros for easy of use.
\end{column}
\end{columns}
\end{frame}


\begin{frame}{\LARGE{Why use \LaTeX ?}\\\Large{Pros and Cons}}
\begin{columns}[t]
\begin{column}{.6\textwidth}
\textbf{\Large{Pros}}
\begin{enumerate}
	{\small
		\item State of the art typesetting 
		\item Templates often offered by journals, conferences, etc. for formatting ease
		\item Complete control over the document
		\item Automation \& cross-referencing
		\item Ease and aesthetics of mathematical equations
		\item Bibliography compilation and citations
		\item ``Large and active user-base with help forums; Free''}
\end{enumerate}
\end{column}
\begin{column}{.5\textwidth}
\textbf{\Large{Cons}}
\begin{enumerate}
	{\small
		\item Difficult to learn (relative to other programs)
		\item Not a WSIWG wordprocessor (e.g., Word)
		\item Difficult when sharing/editing with co-authors}
\end{enumerate}
\end{column}
\end{columns}
\end{frame}



\begin{frame}{\LARGE{Why use \LaTeX?}}
\centering
\fbox{\includegraphics[width=8cm, height=7cm]{tex_example.png}}
\end{frame}

\begin{frame}{\LARGE{Why use \LaTeX?}}
\centering
\includegraphics[width=10cm,height=7cm]{worlatex_light.jpg}
\end{frame}

\begin{frame}{\LARGE{What is \LaTeX?}}
Markup language - converts source text, combined with the markup, into a lovely document (similar to how web pages work).\\~\\
\begin{columns}
\begin{column}{.5\textwidth}
\% Create a document!\\
\textbackslash documentclass\{article\}\\
\textbackslash begin\{document\}\\~\\
Objective 1 is complete!\\~\\
\textbackslash end\{document\}
\end{column}
\begin{column}{.3\textwidth}
\fbox{\includegraphics[width=3cm, height=1cm]{obj1.png}}
\end{column}
\end{columns}
\pause
~\\~\\ \Large{\textbf{Exercise:} Create a simple document.}
\end{frame}

\begin{frame}{}

\centering
\includegraphics[scale=.45]{latex_comic.jpg}
\end{frame}


\section{Layout}
\subsection{Layout}


\begin{frame}{\LARGE{Getting Started}}
LaTeX is a markup language and document preparation system.\\~\\
\begin{center}
	\textbf{What \textit{is} markup?}
\end{center}
Commands to style and format the document. \\~\\
New commands can be created and stored in libraries for reuse (similar to R).\\~\\
Plain text is wrapped within commands, but can be written simply as plain text.

\end{frame}

\begin{frame}{\LARGE{Getting Started}}
The basic parts of a document: Preamble and/or top matter and the body of the document.\\~\\
{\color{red}\textbackslash documentclass}\{{\color{OliveGreen}article}\} \\~\\

{\color{ProcessBlue}\textbf\textbackslash begin\{}{\color{blue}document}{\color{ProcessBlue}\}}\\
Some text\\
{\color{ProcessBlue}\textbf\textbackslash end\{}{\color{blue}document}{\color{ProcessBlue}\}}\\
\end{frame}

\begin{frame}{\LARGE{Getting Started}}
Document classes:
\begin{itemize}
\item \textbf{article:} for scientific articles, short reports, program documentation (the general class for docs)
\item \textbf{proc:} for proceedings based on the article class
\item \textbf{report:} for longer reports containing several chapters, thesis, etc.
\item \textbf{book:} for books
\item \textbf{slides:} presentation slides - uses big sans serif letters
\item \textbf{memoir:}
\item \textbf{letter:} for letters
\item \textbf{beamer:} for presentations
\end{itemize}

\end{frame}

\begin{frame}{\LARGE{Getting Started}}
Document class \underline{options}: {\color{red}\textbackslash documentclass}[ ]\{{\color{OliveGreen}article}\} \\~\\
\begin{itemize}
\item font size
\item paper size
\item alignment and numbering placement of equations (e.g., left or right aligned)
\item column structure
\item to have a standalone title page
\item landscape or portrait
\end{itemize}
{\color{red}\textbackslash documentclass}[12pt, letterpaper, landscape]\{{\color{OliveGreen}article}\}
\end{frame}

\begin{frame}{\LARGE{Preamble}}
Many options! Some basics:
\begin{itemize}
\item specify margins {\color{red}\textbackslash usepackage}[ ]\{{\color{OliveGreen}geometry}\}
\item have document double spaced:\\ {\color{red}\textbackslash usepackage}\{{\color{OliveGreen}setspace}\}\\{\color{red}\textbackslash doublespacing}
\item create title info:\\
{\color{red}\textbackslash author}\{\}\\
{\color{red}\textbackslash title}\{\}\\
{\color{red}\textbackslash date}\{\}
\item load packages for use throughout document:\\
{\color{red}\textbackslash usepackage}[ ]\{{\color{OliveGreen}amsmath, amsfont, color, float, titlesec, ...}\}
\end{itemize}
\end{frame}

\begin{frame}{\LARGE{Document layout and settings}}

{\color{ProcessBlue}\textbf\textbackslash begin\{}{\color{blue}document}{\color{ProcessBlue}\}}\\~\\
\textbf{Titles, Chapters, Sections, etc.}
\begin{itemize}
\normalsize
\item \textbackslash maketitle
\item \textbackslash part\{\dots\}
\item \textbackslash chapter\{\dots\}
\item \textbackslash section\{\dots\}
\item \textbackslash subsection*\{\dots\}
\end{itemize}
{\color{ProcessBlue}\textbf\textbackslash end\{}{\color{blue}document}{\color{ProcessBlue}\}}
\end{frame}

\begin{frame}{Section formatting}
Some section formatting that I commonly use \textbackslash usepackage\{titlesec\}:\\~\\
\textbackslash titleformat\{the command\}\{styling of header text\}\{label \}\{spacing before\}\{spacing after\} \\~\\

\textbackslash titleformat\{\textbackslash textbackslash section\}\{\textbackslash normalfont \textbackslash bfseries\}\{\}\{0pt\}\{\}
\textbackslash titleformat\{\textbackslash subsection\}\{\textbackslash normalfont\textbackslash bfseries\textbackslash itshape\}\{\}\{0pt\}\{\}
\textbackslash titleformat\{\textbackslash subsubsection\}\{\textbackslash normalfont \textbackslash itshape\}\{\}\{0pt\}\{\}

\end{frame}

\begin{frame}{\LARGE{Document layout and settings}}
\begin{tabular}{l|l}
 \textbf{Preamble} & \textbf{Body}\\
\hline
 & \\
\textbackslash documentclass[11pt, fceqn]\{article\} & \textbackslash maketitle\\~\\
\textbackslash usepackage[left=2cm,\dots]\{geometry\} &  \textbackslash section\{\dots\}\\~\\
\textbackslash title\{\dots\} \textbackslash author\{\dots\}  & \textbackslash subsection\{\dots\}\\
\hline
\end{tabular}
~\\[1.5cm]\Large{\textbf{Exercise:} Customize a document with above settings.}
\end{frame}


\begin{frame}{\LARGE{Characters, symbols, and spacing}}
\textbf{Special characters:} \# \$ \% \^{} \& \_ \{ \} \~{} \textbackslash ~\\
\hspace{.5cm} These can all be used if preceded by a \textbackslash \hspace{.2cm}  (e.g., \textbackslash \# or \textbackslash \{ \textbackslash\} ) ~\\~\\
\textbf{Symbols:} command for just about any symbol, all begin with  \textbackslash followed by the command name (e.g., $\cup$ \textbackslash cup, $^{\circ}$ \^{} \{circ\}, $\triangledown$ \textbackslash triangledown, etc.)~\\~\\

\textbf{Spaces:} \textbackslash\textbackslash \hspace{.5mm} is a carriage return, \textbackslash hspace\{1cm\}, \textbackslash vspace\{2in\}, \textbackslash indent, skipped line indicates a new \textparagraph, $\sim$ \textbackslash \textbackslash is a forced carriage return ~, \textbackslash clearpage starts at the top of a new page\\

\end{frame}

\begin{frame}{\LARGE{Characters, symbols, and spacing}}
\centering
\includegraphics[scale=.4]{comic.png}
\end{frame}



\section{Equations}
\subsection{Equations}

\begin{frame}{\LARGE{Equations and Math Mode}}
\textbf{\Large{Equations}}\\~\\

\begin{equation}
CI = \bar{x} \pm z_{\alpha/2}*\biggl(\dfrac{s}{\sqrt{n}}\biggr) = 0.6 \pm 1.65 * \biggl(\dfrac{0.230}{\sqrt{14}}\biggr) = \textbf{[0.4 - 0.8]}
\end{equation}
\textbackslash begin\{equation\}\\~\\
CI = \textbackslash bar\{x\} \textbackslash pm z\_\{\textbackslash alpha/2\}*\textbackslash biggl(\textbackslash dfrac\{s\}\{\textbackslash sqrt\{n\}\}\textbackslash biggr) = 0.6 \textbackslash pm 1.65 * \textbackslash biggl(\textbackslash dfrac\{0.230\}\{\textbackslash sqrt\{14\}\}\textbackslash biggr) = \textbackslash textbf\{[0.4 - 0.8]\}\\~\\
\textbackslash end\{equation\}
\end{frame}

\begin{frame}{\LARGE{Equations and Math Mode}}
\textbf{\Large{Math mode}}\\~\\
\[ \hat{Y_{i}} = \hat{\beta_{0}} + \hat{\beta_{1}}X_{i} + \epsilon_{i} \]~\\~\\

\textbackslash[ \textbackslash hat\{Y\_\{i\}\} = \textbackslash hat\{\textbackslash beta\_\{0\}\} + \textbackslash hat\{\textbackslash beta\_\{1\}\} X\-\{1\} \textbackslash] \\~\\

\Large{Note: no numbering of equations}
\end{frame}

\begin{frame}{\LARGE{Equations and Math Mode}}
\Large{\textbf{Exercise:} Create the following equations using the equation markup for one and general math mode for the other, inside the document you've already created.}
\begin{equation}
N\beta_{0} = \sum\limits_{i=1}^{N}y_{i} -\beta_{1}\sum\limits_{i=1}^{N}x_{i}
\end{equation}~\\
$$var(\bar{Y}) = \dfrac{\gamma_{0}}{n}[1+2 \sum\limits_{k=1}^{n-1} (1-\dfrac{k}{n}) \rho_{k}]$$
\end{frame}

\section{Tables \& Figures}
\subsection{Tables \& Figures}

\begin{frame}{\LARGE{Tables \& Figures}}
\begin{itemize}
\item Tables \& Figures are automatically numbered in \LaTeX
\item Tables can be exported directly from R into \LaTeX, or created within the document itself
\item Figures are imported from a file path name
\item If the source file is updated, it is automatically updated in the document when compiled
\item Both tables and figures can be `labeled' to be cross-referenced in the document
\end{itemize}
\end{frame}

\begin{frame}[fragile]{\LARGE{Table and Figure environment}}
\begin{verbatim}
\usepackage{graphix, float}
\end{verbatim}
Tables and figures are created in the \textit{floating} environment for ease of placement. Pagebreaks aren't allowed within this environment - so LaTeX decides where the image is best suited. You can suppress this, however.\\~\\
\underline{Placements:} \textbf{[t]}: top; \textbf{[b]}: bottom; \textbf{[p]}: put on a separate page without text, only with other figures, tables, and floats; \textbf{[h]}: put the float approximately at the current position; \textbf{[H]} put the float at the current position 
\end{frame}

\begin{frame}{\LARGE{Tables - \LaTeX creation}}
  
\textbackslash begin\{table\}[H]    
    \textbackslash begin\{center\}
  		 \textbackslash begin\{tabular\}\{c$\vert$ l$\vert$ r\} 
  		 
  Centered \& Left \& Right \textbackslash \textbackslash ~\\
  \textbackslash hline ~\\
	A  \& B \& C \textbackslash \textbackslash ~\\
	\textbackslash hline ~\\
	 1 \& 2 \& 3 ~\\
	\textbackslash hline ~\\
  \textbackslash end\{tabular\}
   		\textbackslash end\{center\}
			\textbackslash end\{table\} 
  
    
\begin{table}[H]    
    \begin{center}
  		\begin{tabular}{c|l|r}
  Centered & Left & Right \\
  \hline
	A  & B & C \\
	\hline
	 1 & 2 & 3\\
	\hline
  \end{tabular}
   		\end{center}
			\end{table}

\end{frame}


\begin{frame}{\LARGE{Tables - merging cells}}

\begin{table}[H]
	\begin{tabular}{c|llll|cc}
	\hline
	\textbf{Rank} &	\multicolumn{4}{c|}{\textbf{Fixed Effects}} & \multicolumn{2}{c}{\textbf{Random Effects}}\\
	\hline
	& Y & M & V & T & Trip & Day \\
	\hline\hline
	1 & x & x & x & x & 3038.5 & 0\\
	2 & x & x & x & x &  3038.7 & 0.2\\
	3 & x & x & x & x & 3041.1 & 2.6\\
	4 & x & x & x & x & 3044.0 & 5.5\\
	5 & x & x & x & x  & 3048.5 & 10\\
	6 & x & x & x & x  & 3052.2 & 13.7\\
	7 & x & x & x & x  & 3053.1 & 14.6\\
	8 & x & x & x & x  & 3061.4 & 22.9\\
		\hline
\end{tabular} 
\caption{Some table caption.}
\end{table}
\end{frame}

\begin{frame}{Tables - merging cells}
This can be done using the \textbackslash multicolumn command.\\~\\
\textbackslash multicolumn\{number of columns to merge\}\{alignment and vertical bars\}\{Label/Text\}\\~\\
For examples, instead of just separating your column values with an \&, you specify the following\\~\\
\textbackslash textbf\{Rank\} \& \textbackslash multicolumn\{4\}\{c\}\{ \textbackslash textbf\{Fixed Effects\}\} \& \textbackslash multicolumn\{2\}\{c\}\{ \textbackslash textbf\{Random Effects\}\}\textbackslash \textbackslash \\~\\

NOTE: the number of columns must add up to the number defined for your table. In this example, 7 are specified. The line following this command will revert back to the table defaults.
\end{frame}


\begin{frame}{Tables - merging cells}

{\Large \textbf{Exercise:} Refer back to the table we previously made. Try to merge the header row into one single cell, with a new label.}\\
\hrulefill ~\\~\\
\begin{table}[H]    
	\begin{center}
		\begin{tabular}{c|l|r}
			\multicolumn{3}{c}{Merged cells} \\
			\hline
			A  & B & C \\
			\hline
			1 & 2 & 3\\
			\hline
		\end{tabular}
	\end{center}
\end{table}
\end{frame}

\begin{frame}{\LARGE{Tables - Export from R}}
In R:\\~\\

library(xtable)\\~\\

tab=xtable(tab1, caption= `Correlation matrix', 
align=c(`c',$\vert$ `c',$\vert$ `c',), digits=3) \\~\\

print(tab,file=``MyTexFile.tex'',append=T,table.placement = ``H'', caption.placement=``top'',hline.after=seq(from=-1, to=nrow(tab), by=1), include.rownames=F) 

\end{frame}

\begin{frame}{\LARGE{Tables - Export from R}}

{\Large \textbf{Exercise:} Attempt to create a table in R, and export into the document you are working on.}\\
\hrulefill ~\\~\\

Simple R table creation:\\~\\

df $<$- data.frame("Year"=seq(2010,2017), "Site"=LETTERS[1:8], "Value"=rnorm(8,5,2) )

\end{frame}

\begin{frame}{\LARGE{Figures - Importing}}
 \textbackslash begin\{figure\}[H]\\~\\
 \textbackslash includegraphics[width=8cm, height=6cm]\{Barplot.pdf\}\\~\\
 
 
     \textbackslash caption\{A barplot \dots\}~\\
            \textbackslash label\{fig:barplot\}~\\
         \textbackslash end\{figure\} ~\\
         
\end{frame}         

\begin{frame}{\LARGE{Figures}}
\Large{\textbf{Exercise:} Insert one or more figures into your document.}\\
\hrulefill ~\\~\\

\normalsize

 \textbackslash begin\{figure\}[H]\\
 \textbackslash includegraphics[width=8cm, height=6cm]\{Barplot.pdf\}\\
     \textbackslash caption\{A barplot \dots\}~\\
            \textbackslash label\{fig:barplot\}~\\
         \textbackslash end\{figure\} ~\\

\end{frame}

\begin{frame}[fragile]{Aligning multiple figures}
Controlling placement - can be done when creating figures (e.g., in R) or aligning them in LaTeX. \textbf{Columns} are one way to do it in LaTeX.

\begin{verbatim}
\begin{columns}
\begin{column}{.5\textwidth}
\includegraphics[width=3cm, height=3.5cm]{FigA.png}
\end{column}
\begin{column}{.5\textwidth}
\includegraphics[width=3cm, height=3.5cm]{FigB.png}
\end{column}
\end{columns}
\end{verbatim} 

\end{frame}

\begin{frame}[fragile]{Multiple figures}

{\small
\begin{verbatim}
\begin{figure}[H]
	\flushleft
	\subfloat{}
	{\includegraphics[width=.3\textwidth, height=5cm]{FigA.png}}
	\subfloat{}
	{\includegraphics[width=.3\textwidth, height=5cm]{FigB.png}}
	\subfloat{}
	{\includegraphics[width=.3\textwidth, height=5cm]{FigC.png}}
	\subfloat{}
	{\includegraphics[width=.3\textwidth, height=5cm]{FigD.png}}
	\subfloat{}
	{\includegraphics[width=.3\textwidth, height=5cm]{FigE.png}}
	\subfloat{}
	{\includegraphics[width=.3\textwidth, height=5cm]{FigF.png}}
	\caption{A series of plots.} 
	\label{fig:series}
\end{figure}
\end{verbatim}
}
\end{frame}

\begin{frame}{\LARGE{Tables \& Figures}}

{\Large Referencing tables and figures within the text}

 \textbackslash begin\{figure\}[H]\\~\\
 \textbackslash includegraphics[width=8cm, height=6cm]\{Barplot.pdf\}\\~\\
 
 
     \textbackslash caption\{A barplot \dots\}~\\
            \textbackslash label\{fig:barplot\}~\\
         \textbackslash end\{figure\} ~\\~\\

The results of this study are shown in Figure \textbackslash ref\{fig:barplot\}.\\~\\

What appears is:\\
The results of this study are shown in Figure 1.
\end{frame}

\begin{frame}{\LARGE{Cross-referencing}}

{\Large \textbf{Exercise:} Reference a table and figure in your document within the text, using cross-referencing.}\\
\hrulefill ~\\

\end{frame}

\begin{frame}{\LARGE{Objectives}}
\begin{itemize}
\Large{
\item Create a document! \checkmark
\item Explore document layout \checkmark
\item Create an equation \checkmark
\item Insert a figure \& a table \checkmark
\item BibTex and References
\item Explore formatting and basic markup tools
\item Sweave
\item Beamer}
\begin{center}

\end{center}
\end{itemize}
\end{frame}

\section{BibTex}
\subsection{BibTex}

\begin{frame}{\LARGE{BibTex and References}}
\begin{itemize}
\Large{
\item Maintain a list of references (text editor or Mendeley?)
\item Export references as .BibTex format (i.e., .bib file)
\item Configure LaTeX for BibTex (should be a one time thing)
\item Cite references in .tex file using label (from .bib file)
\item Reference .bib in LaTeX file}
\end{itemize}
\end{frame}


\begin{frame}{\LARGE{BibTex and References}}
\begin{center}
 \includegraphics[width=8cm, height=6cm]{BibTex.png}
 \end{center}
\end{frame}

\begin{frame}{\LARGE{BibTex and References}}

\ldots Identification of reliable indicators (i.e., quantifiable metrics responsive to change) of ecosystem dynamics as they relate to the state space \textbackslash citep\{walker\_2004\} is an important area of research because the transition to a new basin of attraction may represent a unidirectional shift into an alternate, long-term stable state, a condition commonly referred to as a regime shift \textbackslash citep\{may\_1977, lluch\_belda\_1989, scheffer\_2001, kuehn\_2011, drake\_2013\}.\\~\\ 

\textbackslash bibliographystyle\{ecology\}\\
\textbackslash bibliography\{oneida\}
\end{frame}


\begin{frame}{\LARGE{BibTex and References}}
There are different reference packages. I use natbib most often. Below are some customizations allowed with that package. You might want to explore some other packages, depending on your preferences.\\~\\

\textbackslash documentclass[article]\\
\textbackslash usepackage[round]\{natbib\}\\~\\
\textbackslash begin\{document\}\\~\\
...\\
\textbackslash bibliographystyle\{apalike\}\\
\textbackslash bibliography\{References\}\\

\textbackslash end\{document\}
\end{frame}

\begin{frame}{\LARGE{BibTex and References}}
Some options:\\
\textbackslash usepackage[square,numbers]{natbib}\\
\textbackslash usepackage[super]{natbib}\\
\textbackslash usepackage[curly,numbers, super]{natbib}\\~\\

\textbackslash bibliographystyle\{plain\}\\
\textbackslash bibliographystyle\{unsrt\}\\
\end{frame}

\begin{frame}{\LARGE{Inline citations}}
some text \textbackslash cite\{maunder2004standardizing\}. = some text Maunder (2004).\\
some text \textbackslash citep\{maunder2004standardizing\}.  = some text (Maunder, 2004).\\
some text \textbackslash citealp\{maunder2004standardizing\}.  = some text Maunder, 2004.\\
some text \textbackslash citealt\{maunder2004standardizing\}.  = some text Maunder 2004.\\
some text \textbackslash citeauthor\{maunder2004standardizing\}.  = some text Maunder.\\
some text \textbackslash citeyear\{maunder2004standardizing\}.  = some text 2004.\\

\end{frame}


\section{Misc Markup}
\subsection{Misc Markup}

\begin{frame}{\LARGE{Text formatting}}
Hot keys used in Word sometimes work!\\
Ctrl B for \textbf{bold} creates: \textbackslash textbf\{\}\\
Ctrl I for \textit{italics} creates: \textbackslash textit\{\}\\~\\

Font sizing: \textbackslash LARGE \{ \Large{LARGE} \normalsize \}\\
E.g., \textbackslash footnotesize, \textbackslash small, \textbackslash normalsize, \textbackslash Large, \textbackslash Huge \\~\\

Font color: \textbackslash textcolor\{blue\}\{\} \textcolor{blue}{Blue Text}
\end{frame}

\begin{frame}{\LARGE{Lists}}

\textbackslash begin \{itemize\}\\


\begin{itemize}
\item \textbackslash item A regular item
\item[$\blacktriangleright$] \textbackslash item[\$\textbackslash blacktriangleright\$] A fancy item
\end{itemize}
\textbackslash end \{itemize\}~\\~\\

\textbackslash begin \{enumerate\}\\
\begin{enumerate}
\item \textbackslash item One item
\item \textbackslash item Two items
\end{enumerate}
\textbackslash end \{enumerate\}
\end{frame}

\begin{frame}{\LARGE{Misc Markup}}
\Large{\textbf{Exercise:} Create lists, and nested lists with special text formatting.}\\
\hrulefill ~\\
\normalsize
\textbackslash begin \{itemize\}\\
\begin{itemize}
\item \textbackslash item A regular item
\item[$\blacktriangleright$] \textbackslash item[\$\textbackslash blacktriangleright\$] A fancy item
\end{itemize}
\textbackslash end \{itemize\}~\\~\\

Font sizing: \textbackslash LARGE \{ \Large{LARGE} \normalsize \}\\
E.g., \textbackslash footnotesize, \textbackslash small, \textbackslash normalsize, \textbackslash Large, \textbackslash Huge \\~\\

Font color: \textbackslash textcolor\{blue\}\{\} \textcolor{blue}{Blue Text}
\end{frame}



\begin{frame}{\LARGE{Packages \& Google}}
\begin{itemize}
\item Just like R, there are many tools for \LaTeX that can be found in specific packages.
\item Usually I find this out from Google searches
\item But, it \textit{usually} doesn't hurt to include several packages in a template
\item Errors can sometimes be cryptic...
\item You can create new commands - similar to creating functions in R \textbackslash newcommand\{ \dots\}
\item Some packages I frequently use: \textbackslash usepackage\{amsmath, amssymb, amsfonts, cite, amsthm, mathtools, color, float, lineno, subfig, enumitem, verbatim, framed, textcmds, libertine\}
\end{itemize}
\end{frame}

\begin{frame}{\LARGE{References}}
Some potentially useful references:\\~\\
All things \LaTeX: 
\textcolor{blue}{http://en.wikibooks.org/wiki/LaTeX}~\\~\\ 


Document class options: \textcolor{blue}{http://www.nada.kth.se/$\sim$carsten/latex/class.html}~\\~\\

Symbols: 
\textcolor{blue}{http://www.andy-roberts.net/res/writing/latex/symbols.pdf}
\end{frame}

\section{knitr}
\subsection{knitr}

\begin{frame}{knitr/Sweave}
\begin{center}
\includegraphics[scale=0.4]{knit.png}
\end{center}
\end{frame}

\section{Beamer}
\subsection{Beamer}

\begin{frame}[fragile]{Beamer}
\begin{verbatim}

\documentclass[xcolor=dvipsnames]{beamer}

\usetheme{Singapore}
\usepackage{color, graphicx, tikz, verbatim} 

\title{\Huge{\textbf{\LaTeX\\ Document Preparation}}}
\author{\textbf{\large{Tiffany Cunningham}}}
\date{\today}

\begin{document}

\begin{frame}
\titlepage
%end{frame}

\end{verbatim}

\end{frame}

\begin{frame}[fragile]{Beamer customization}
There are several standard themes that you can easily call to modify the appearance of your presentations. Here are a few:\\
\begin{verbatim}
http://deic.uab.es/~iblanes/beamer_gallery/
\end{verbatim}
Some themes:
\begin{itemize}
	\item Bergen
	\item boxes
	\item Dresden
	\item Malmoe
\end{itemize}
Some color themes:
\begin{itemize}
	\item albatross
	\item crane
	\item lily
\end{itemize}	
\end{frame}

\begin{frame}[fragile]{Beamer customization}
\begin{verbatim}
\usetheme{Singapore}
\usecolortheme{lily}

# Or define your own color scheme
%\usecolortheme[RGB={194,91,91}]{structure}

# Remove navigation symbols
\setbeamertemplate{navigation symbols}{}

\end{verbatim}
\end{frame}


\begin{frame}{Beamer customization}
There is really endless customization you can do. If you like animation... you might be stuck with PowerPoint, otherwise, this might offer a nice alternative. \\~\\

Can also make posters... I'll leave it here. The basics are pretty straightforward. Google will help you with the rest.
\end{frame}

\section{Wrap Up}
\subsection{Wrap Up}
\begin{frame}
\includegraphics[scale=0.35]{end.jpg}
\end{frame}

\end{document}